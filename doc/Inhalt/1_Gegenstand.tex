\section{Warum die Industrie 4.0 Anwendungen benötigt}

Die Produktion ist teuer und Maschinen arbeiten nicht zuverlässig genug. Erhöhte Energieverbrauchswerte und längere Fertigungszeiten der Produktionsanalgen sind auf den ersten Blick nicht zu erkennen. Die Ursache, beispielsweise eine poröse Isolierung eines Härteofens, wird in der Regel zu spät erkannt. Unnötige und teure Ausfälle der Produktion haben meist schwerwiegende finanzielle Folgen.

Ein Produktionsausfall ist bis dato nicht vorhersehbar, eine Reparatur der Anlage muss zunächst beauftragt werden. Dabei geht wertvolle Zeit verloren und die Kosten vervielfachen sich. Im schlimmsten Fall drohen Vertragsstrafen, da Kundenzusagen nicht mehr einzuhalten sind. Noch gravierender ist ein Szenario, in dem der Ausfall das in der Anlage befindliche Produkt zerstört. Das produzierende Gewerbe muss daher besser aufgestellt sein, um proaktiv agieren können. 


\subsection{Die aktuelle Situation}
Die Nachfrage nach einer entsprechenden Lösung ist groß. Als Beispiel soll eine Lohnhärterei dienen, die mehrere Wärmebehandlungsanlagen betreibt und unter hohem zeitlichen Druck steht. Im Durchschnitt fallen dort 1-2 Anlagen pro Jahr aus, was schnell Kosten im hohen fünfstelligen Bereich verursacht und Folgeschäden wie Imageverlust nach sich zieht. Dabei gibt es gerade in diesem Bereich bereits viele protokollierte Daten, die nicht genutzt werden. Das ist kein Einzelfall, wie die Studie von SAS belegt (vgl. \cite{SasForsa}, S. 8). Viele Unternehmen nutzen Ihre Daten nicht oder nur teilweise.

Verschiedene, vor allem große Anbieter wie SAS oder Telekom sind bereits auf dem Markt vertreten und bieten meist Standardlösungen. Wir hingehen wollen uns auf eine bestimmte Branche fokussieren und IT-Dienstleistung in dieser individuell anbieten. Für das Vorhersagen bestimmter Ereignisse benötigen wir Sensorik, bei der uns aktuell das nötige Know-how fehlt. Daher streben wir eine Kooperation mit einem Anlagenhersteller an, der über das notwendige Spezialwissen verfügt. Die Kombination aus IT-Kompetenz, unser Erfahrung aus den Bereichen Monitoring und Predictive Maintenance und dem verfahrenstechnischen Know-How erlaubt es uns dem Kunden eine maßgeschneiderte Lösung zu bieten.

Uns ist bekannt, dass ein Teil der Unternehmer keinen Mehrwert durch Industrie 4.0 Anwendungen sieht (vgl. \cite{SasForsa}, S. 5). Jedoch gibt es viele Gründe, die für den Einsatz sprechen. Anhand einer Wirtschaftlichkeitsrechnung lässt sich aufzeigen, dass es sich lohnt zu investieren. Das sehen laut der Forsa-Umfrage von SAS 81 Prozent genauso, die eine Investition für richtig halten (vgl. \cite{SasForsa}, S. 9).

Bis 2020 "`will die deutsche Industrie 40 Milliarden Euro pro Jahr in Anwendungen von Industrie 4.0 investieren"' (vgl. \cite{IndustrieHohesPotenzial}). Die Investitionsbereitschaft zeigt die große Nachfrage im Bereich der Industrie 4.0 Anwendungen deutlich auf. 






