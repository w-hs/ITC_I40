\section{Management-Summary}

In unserem Dienstleistungsportfolio zur Industrie 4.0 geht es um die Unzuverlässigkeit von Industrieanlagen und deren unbemerkte Zustandsverschlechterung. Da Produktionsausfälle unnötig Zeit in Anspruch nehmen und hohe Kosten verursachen, haben wir eine IT-Dienstleistung zur Lösung des Problems erarbeitet. Wir bieten den Betreibern von Industrieanlagen Beratung und Projektkoordination bei der Einführung unserer Predictive-Maintenance-Lösung.

Ein solches System ist wichtig um eine hohe Anlagenverfügbarkeit zu garantieren, Lieferzeiten und Qualitätsversprechen einzuhalten, Kundenzufriedenheit zu erhöhen und Kosten zu reduzieren. 81 Prozent der deutschen Industrieunternehmen haben die Wichtigkeit erkannt (vgl. \cite{SasForsa}, S. 9) und wollen bis 2020 jährlich 40 Milliarden Euro investieren (vgl. \cite{IndustrieHohesPotenzial}).

Am Markt sind bereits einige Anbieter mit Standardlösungen vertreten. Unsere Lösung ist hingehen individuell und kundennah. Gemeinsam mit unserem Kunden erarbeiten wir eine perfekt zugeschnittene Lösung und koordinieren die Einführung unserer Predictive-Maintenance-Lösung. Durch eine Kooperation mit einem ausgewählten Anlagenbauer erreichen wir potentielle Kunden. Im Gegenzug profitiert unser Kooperationspartner von einer verbesserten Kundenbindung. Durch die Möglichkeit den Anlagenbetreibern proaktiv anzusprechen und auf bevorstehende Ereignisse aufmerksam zu machen revolutioniert er den Aftermarket.

Wir benötigen ein Startkapital von 450.000 Euro, um im zweiten Quartal des zweiten Jahres die Gewinnschwelle zu erreichen. Mit Abschluss des Dreijahresplans erwarten wir einen Return of Investment von über 30 Prozent.

Mit unserem Know-how und der Erfahrung aus den Bereichen Monitoring und Predictive Maintenance sind wir der bestmögliche Partner eines Anlagenbauers. Unsere Flexibilität und unser Spirit machen uns auch für den Endkunden interessant. Trotz guter Chancen sind wir aufgrund unserer Vorstellung zu leben und dem finanziellen Risiko nicht bereit eine Selbstständigkeit aufzubauen. Daher würden wir unsere Idee am liebsten gemeinsam mit einem erfahrenen IT-Dienstleiter implementieren.
