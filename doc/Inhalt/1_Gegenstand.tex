\section{Warum braucht die Welt Dienstleistung für Industrie 4.0?}

- Titel: Warum braucht die Industrie unsere Dienstleistung?
- Schritt 1: Gegenstand festlegen und beschreiben
- Was genau ist das Problem
- Gegenstand festlegen und beschreiben
- ...

Schritt 1: Gegenstand festlegen und beschreiben

Fragen die hier gestellt werden

Was genau ist das Problem?
Maschinen arbeiten unzuverlässig und die Produktion ist teuer.
Zustandsänderungen schwierig zu erkennen
Schlechte Isolierung führt zu Wärmeverlust in Härteöfen. Dadurch steigt der Energieverbrauch. Diese Erkenntnis erfordert eine Wartung.

Warum ist hier ein Problem?
Kostet mehr als es kosten müsste
Wettbewerbsnachteile
Wir werden von Fehlern überrascht.

Weshalb muss es gelöst werden?
Um wettbewerbsfähig zu bleiben
Proaktiv agieren und nicht erst auf Fehler warten

Welche Teile sind bekannt, welche unbekannt?
Bekannt
Die Unternehmen wollen investieren, Geld ist da
Der Markt ist nicht gesättigt
Beratung kann gut verkauft werden, da wir bewerten können was an Industrie 4.0 möglich und sinnvoll ist.
Unbekannt
Wettbewerbslage (Wir wissen nicht genau, wer sich bereits intensiv mit dem Thema beschäftigt)
Technik bzw. Sensorik
Gesetzeslage / Normen / Datenschutz
Akzeptanz für Berater in Firmen
Wollen Firmen überhaupt in die Beratung investieren, oder es lieber selbst machen?
Eventuell ist die Sensorik teurer als die Reparatur
Siehe Studie: SAS Deutschland 2013 “Gründe gegen die Auswertung von Maschinendaten”

Welche Zusammenhänge?
Beratung und mögliche (vielleicht ja unsere eigenen) Produkte
Hersteller (Sensorik, Maschinenbauer) und uns Beratern
Für dieses Szenario können wir folgende Sensorik empfehlen

Wie kann das Problem eingegrenzt werden?
Wartungszyklen verkürzen
Indem die Maschinen überwacht werden
Sensoren und Messdaten auswerten

Ist das Problem quantitativ abbildbar?
Wieviel wollen die Unternehmen investieren?
Deutsche Industrie in Summe 40 Mrd. jährlich (s. PWC “2”)
Wieviele Unternehmen wollen investieren?
81% halten Investitionen für richtig (s. Einschätzung zur Bedeutung der Analyse von Maschinendaten, SAS)
Welche Unternehmen: Anzahl Maschinen?
Durchschnittliche Ausfallraten und dadurch verursachte Kosten?
Wie viele solcher Systeme gibt es / sind bereits im Einsatz?
75% nutzen bereits teilweise solche Systeme (s. Auswertung von Maschinendaten, SAS)
viele dieser Systeme nutzen nur Teile der verfügbaren Daten (s. Anteil der ausgewerteten Daten, SAS) 

In welche Teilprobleme kann es zerlegt werden?
Teilprobleme haben wir bereits bei der Frage nach dem eigentlichen Problem beschrieben.

Wer ist davon betroffen?
Das produzierende Gewerbe im Mittelstand
Wir, da wir Beratung in diesem Bereich anbieten wollen
Vielleicht die Arbeitnehmer, da diese auf Dauer zu teuer sind
