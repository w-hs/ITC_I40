\section{Warum die Industrie 4.0 Anwendungen benötigt}

Die Produktion ist teuer und Maschinen arbeiten unzuverlässig. Erhöhte Energieverbrauchswerte und längere Fertigungszeiten der Produktionsanalgen sind auf den ersten Blick nicht zu erkennen. Die Ursache, beispielsweise eine poröse Isolierung eines Härteofens, wird zu spät erkannt und es kommt zu einem unnötigen und teurem Ausfall der Produktion.

Ein Produktionsausfall ist bis dato nicht vorhersehbar, eine Reparatur der Anlage muss zunächst beauftragt werden. Dabei geht wertvolle Zeit verloren und die Kosten vervielfachen sich. Im schlimmsten Fall drohen Vertragsstrafen, da Kundenzusagen nicht mehr einzuhalten sind. Das produzierende Gewerbe muss daher proaktiv auf Ereignisse agieren können. 


\subsection{Die aktuelle Situation}
Die Nachfrage nach einer entsprechenden Lösung ist groß. Einige Anbieter sind bereits auf dem Markt und bieten Standardlösungen. Wir hingehen wollen uns auf eine bestimmte Branche fokussieren und IT-Dienstleistung individuell anbieten. Für das Vorhersagen bestimmter Ereignisse benötigen wir Sensorik, bei der uns aktuell das nötige Know-how fehlt. Daher streben wir eine Kooperation mit einem Anlagenhersteller, der über das notwendige Spezialwissen verfügt, an. Zusammen mit unserem informationstechnischem Wissen können wir dem Kunden so eine maßgeschneiderte Lösung bieten.

Uns ist bekannt, dass ein Teil der Unternehmer keinen Mehrwert durch Industrie 4.0 Anwendungen sehen (vgl. \cite{SasForsa}, S. 5). Jedoch gibt es viele Gründe die für den Einsatz sprechen. Anhand einer Wirtschaftlichkeitsrechnung lässt sich aufzeigen, dass es sich lohnt zu investieren.

Bis 2020 "`will die deutsche Industrie 40 Milliarden Euro pro Jahr in Anwendungen von Industrie 4.0 investieren"' (vgl. \cite{IndustrieHohesPotenzial}). Die große Investitionsbereitschaft zeigt die große Nachfrage im Bereich der Industrie 4.0 Anwendungen. 

\subsection{Eine mögliche Idee}


Wie kann das Problem eingegrenzt werden?
Wartungszyklen verkürzen
Indem die Maschinen überwacht werden
Sensoren und Messdaten auswerten

Ist das Problem quantitativ abbildbar?
Wieviel wollen die Unternehmen investieren?
Deutsche Industrie in Summe 40 Mrd. jährlich (s. PWC “2”)
Wieviele Unternehmen wollen investieren?
81% halten Investitionen für richtig (s. Einschätzung zur Bedeutung der Analyse von Maschinendaten, SAS)
Welche Unternehmen: Anzahl Maschinen?
Durchschnittliche Ausfallraten und dadurch verursachte Kosten?
Wie viele solcher Systeme gibt es / sind bereits im Einsatz?
75% nutzen bereits teilweise solche Systeme (s. Auswertung von Maschinendaten, SAS)
viele dieser Systeme nutzen nur Teile der verfügbaren Daten (s. Anteil der ausgewerteten Daten, SAS) 

halbes jahr, 1 unplanmäßiger asufall / wartung => 80 - 100 T. Euro

